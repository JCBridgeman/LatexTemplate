% This line sets the project root file.
% !TEX root = ../../subsystemBeyondPauli.tex
% !TeX spellcheck = en_US

\section{Pauli subsystem codes}\label{app:pauliSubsystemCodes}

For completeness, in this section we will review the definition of a Pauli subsystem code following \onlinecite{Poulin2005,Bombin2010a}.

Setting notation:
Let $G$ be a group.
\begin{subalign}
	\text{Center: }     &  & \cent{G}:=  & \set{g\in G}{ghg^{-1}=h,\,\forall h\in G},                            \\
	%	\text{Centralizer: } &  & \centralizer{H}{G}:= & \set{g\in G}{ghg^{-1}=h,\,\forall h\in H},    \\
	%	\text{Normalizer: }  &  & \normalizer{H}{G}:=  & \set{g\in G}{ghg^{-1}\in H,\,\forall h\in H}, \\
	\text{Rank: }       &  & \rk{G}:=    & \text{size of minimal generating set},                                \\
	\text{Order: }      &  & \order{G}:= & \text{number of elements of $G$},                                     \\
	\text{Commutator: } &  & [g,h]:=     & ghg^{-1}h^{-1}                             & \text{for $g,\,h\in G$}.
\end{subalign}

We begin with an abstract definition of the Pauli group for $n$ qudits of dimension $q$.
Later we will act on the vector space $(\mathbb{C}^{q})^{\otimes n}$ using the standard representation.

\begin{definition}[Pauli group]
	Let $n$ and $q$ be positive integers.
	The \define{$(q,n)$-Pauli group} is defined by
	\begin{align}
		\Pauli{q}{n}:= &
		\begin{cases}
			\grouppresentation{\alpha_{q},X_1,\ldots,X_n,Z_1,\ldots,Z_n}{X_i^q=Z_i^q=\alpha_{q}^{2q}=\alpha_{q}^{2}\comm{X_i,Z_i}=1} & \text{$q$ even} \\
			\grouppresentation{\alpha_{q},X_1,\ldots,X_n,Z_1,\ldots,X_n}{X_i^q=Z_i^q=\alpha_{q}^q=\alpha_{q}\comm{X_i,Z_i}=1}        & \text{$q$ odd.}
		\end{cases}
	\end{align}
	The group has order and rank
	\begin{subalign}
		\order{\Pauli{q}{n}}= &
		\begin{cases}
			2q^{2n+1} & \text{$q$ even} \\
			q^{2n+1}  & \text{$q$ odd},
		\end{cases} \\
		\rk{\Pauli{q}{n}} =   &
		\begin{cases}
			2n+1 & \text{$q$ even} \\
			2n   & \text{$q$ odd}.
		\end{cases}
	\end{subalign}
	Each element of $\Pauli{q}{n}$ can be expressed uniquely as
	\begin{align}
		g & =\alpha_{q}^{a}X_1^{x_1}X_2^{x_2}\ldots X_n^{x_n}Z_1^{z_1}Z_2^{z_2}\ldots Z_n^{z_n},\label{eqn:pauliGroupWord}
	\end{align}
	for some $0\leq x_i,z_i<q$, and
	\begin{align}
		0\leq a<\begin{cases}
			        2q & \text{ $q$ even} \\
			        q  & \text{ $q$ odd}.
		        \end{cases}
	\end{align}
	The center of $\Pauli{q}{n}$ is
	\begin{align}
		\cent{\Pauli{q}{n}}= & \grouppresentation{\alpha_{q}}{}\cong
		\begin{cases}
			\ZZ{2q} & \text{$q$ even} \\
			\ZZ{q}  & \text{$q$ odd.}
		\end{cases}
	\end{align}
\end{definition}
%
\begin{definition}[Stabilizer group]
	A \define{stabilizer group} is an abelian subgroup $\Stab\subgp\Pauli{q}{n}$ such that
	\begin{align}
		\Stab\cap\cent{\Pauli{}{}}= & \{1\}.
	\end{align}
	Following \onlinecite{Bombin2010a}, a stabilizer group can be specified by supplying a pair $(\phi, s)$, where $\phi$ is an automorphism of $\Pauli{q}{n}$, and $s\leq n$ is the rank of $\Stab$.
	The $s$ generators of $\Stab$ are the images of $X_1,\ldots,X_s\in\Pauli{q}{n}$
	\begin{align}
		\Stab:= & \grouppresentation{\phi(X_1),\phi(X_2),\ldots,\phi(X_s)}{}.
	\end{align}
	The group has order and rank
	\begin{subalign}
		\order{\Stab}= & q^s, \\
		\rk{\Stab} =   & s.
	\end{subalign}
	We remark that there is a large redundancy in this description, with many choices of $\phi$ leading to the same $\Stab$.
\end{definition}
%
\begin{definition}[Gauge group]
	A \define{gauge group} $\G\subgp\Pauli{q}{n}$ is specified by a triple $(\phi,s,r)$, where $\phi$ is an automorphism of $\Pauli{q}{n}$ and the non-negative $s$, $r$ obey $s+r\leq n$.
	Provided with this data, the gauge group is defined as
	\begin{subalign}
		\G:= & \grouppresentation{\alpha_{q},\phi(X_1),\phi(X_2),\ldots,\phi(X_s),\phi(X_{s+1},\ldots,\phi(X_{s+r}),\phi(Z_{s+1}),\ldots,\phi(Z_{s+r})}{} \\
		=    & \grouppresentation{\alpha_{q},\Stab,\phi(X_{s+1},\ldots,\phi(X_{s+r}),\phi(Z_{s+1}),\ldots,\phi(Z_{s+r})}{},
	\end{subalign}
	where $\Stab = \grouppresentation{\phi(X_1),\phi(X_2),\ldots,\phi(X_s)}{}$ is the stabilizer subgroup.

	The group $\G$ has order and rank
	\begin{subalign}
		\order{\G}= &
		\begin{cases}
			2q^{s+2r+1} & \text{$q$ even} \\
			q^{s+2r+1}  & \text{$q$ odd},
		\end{cases} \\
		\rk{\G} =   &
		\begin{cases}
			s+2r+1 & \text{$q$ even} \\
			s+2r   & \text{$q$ odd}.
		\end{cases}
	\end{subalign}
\end{definition}

\begin{definition}[Bare logical group]
	Given a gauge group $\G$ defined by the triple $(\phi,s,r)$, the \define{bare logical group} is
	\begin{align}
		\bareLog & = \grouppresentation{\alpha_{q},\phi(X_{s+r+1}),\phi(Z_{s+r+1}),\phi(X_{s+r+2}),\phi(Z_{s+r+2}),\ldots,\phi(X_{s+r+k}),\phi(Z_{s+r+k})}{},
	\end{align}
	where $s+r+k=n$.
	The minimal length of a non-trivial (not only containing powers of $\alpha_{q}$) word $W$ in $\bareLog$ is the \define{bare distance} $\bareDistance$.
	Equivalently, $W$ is non-trivial if $\phi^{-1}(W)$ contains at least one element of $\{X_{s+r+1},Z_{s+r+1},X_{s+r+2},Z_{s+r+2},\ldots,X_{s+r+k},Z_{s+r+k}\}$ when written in standard form \cref{eqn:pauliGroupWord}.
\end{definition}

\begin{definition}[Dressed logical group]
	Given a gauge group $\G$ defined by the triple $(\phi,s,r)$, the dressed logical group is
	\begin{align}
		\dressedLog = \grouppresentation{\alpha_{q},\G,\phi(X_{s+r+1}),\phi(Z_{s+r+1}),\phi(X_{s+r+2}),\phi(Z_{s+r+2}),\ldots,\phi(X_{s+r+k}),\phi(Z_{s+r+k})}{}.
	\end{align}
	The minimal length of a non-trivial word in $\dressedLog$ is the \define{dressed distance} $\dressedDistance$.
\end{definition}

\begin{definition}[Pauli subsystem code]
	Given a gauge group $\G$ acting on $n$ physical qudits by the standard representation, the stabilizer group partitions the Hilbert space into two subspaces
	\begin{align}
		(\mathbb{C}^q)^n\cong \C\oplus\C^{\perp},
	\end{align}
	where $\C$ is the subspace on which $\Stab$ acts as $+1$, with dimension $q^{n-s}$.
	In a Pauli subsystem code, the code space $\C$ is further decomposed into
	\begin{align}
		\C\cong \mathcal{A}\otimes\mathcal{B},
	\end{align}
	where $\mathcal{A}$ is the logical subsystem, and $\mathcal{B}$ is called the gauge space.
	The gauge group $\G$ acts trivially on the $q^k$-dimensional subsystem $\mathcal{A}$, while $\bareLog\cong\Pauli{q}{k}$ acts as the logical group on this system.
	The remaining $q^r$-dimensional subsystem $\mathcal{B}$ has non-trivial $\G$ action.
	%
	Pragmatically, the gauge space is a `junk' space that can be used to bypass no-go theorems for pure stabilizer codes.
	%
	A Pauli subsystem code on $n$ physical qudits, each of dimension $q$, which encodes $k$ logical qudits with bare distance $\bareDistance$ and dressed distance $\dressedDistance$ has parameters
	\begin{align}
		\codeparameters{n,k,\bareDistance,\dressedDistance}{q}.
	\end{align}
	It is often convenient to express the various groups in terms of a generating set.
	The number of encoded qudits $k$ can be calculated from the groups by
	\begin{align}
		k = & n-(s+r) \\
		=   &
		\begin{cases}
			n-\frac{\rk{\G}+\rk{\S}-1}{2} & \text{$q$ even} \\
			n-\frac{\rk{\G}+\rk{\S}}{2}   & \text{$q$ odd}.
		\end{cases}
	\end{align}

	Typically, a Pauli subsystem code is specified by providing an (over-complete) generating set for the gauge group $\G$, usually omitting mention of $\alpha_{q}$.
	The stabilizer subgroup then consists of the center of the group defined by the given generators, ignoring phases.
	We follow this tradition in the main text.
\end{definition}